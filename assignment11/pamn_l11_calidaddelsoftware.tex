\documentclass{article}
\usepackage{graphicx}
\usepackage{float}
\usepackage[spanish]{babel}
\usepackage{hyperref}
\usepackage{csquotes}
\graphicspath{ {img/} }
\setlength{\parskip}{\baselineskip}

\title{Calidad en el Software \\[3ex] \small PAMN - Programación de Aplicaciónes Moviles Nativas}

\author{Chamil José Cruz Razeq}

\begin{document}
    \maketitle
    \thispagestyle{empty}
    \newpage

    \section{Introducción}
        Todos los informes sobre las tareas propuestas se encuentran disponibles en el
         siguiente repositorio de \href{https://github.com/chamilstudy/ulpgc_pamn_assigments}{GitHub}.

        Se ha realizado una lista de control basados en los criterios de calidad discutidos en clase.
    
    \section{A Nivel Funcional}

    \begin{itemize}
        \item \textbf{Exactitud de los datos}: La aplicación debe proporcionar información precisa y actualizada.
        \item \textbf{Rendimiento}: La aplicación funcionar de manera rápida y eficaz, sin degradar la experiencia del usuario.
        \item \textbf{Transparencia}: La aplicación debe informar al usuario sobre el manejo de su información.
        \item \textbf{Facilidad de uso}: La aplicación debe ser fácil de usar y navegar.
    \end{itemize}

    \section{A Nivel No-Funcional}

    \begin{itemize}
        \item \textbf{Disponibilidad}: La aplicación debe estar disponible para los usuarios en todo momento.
        \item \textbf{Escalabilidad}: La aplicación debe ser capaz de manejar un aumento en el número de usuarios o en la cantidad de datos.
        \item \textbf{Mantenibilidad}: La aplicación debe ser fácil de mantener y actualizar.
        \item \textbf{Portabilidad}: La aplicación debe ser compatible con diferentes plataformas y dispositivos (en nuestro caso contemplamos solo
            la aplicación nativa, pero podrían expandirse algunas funcionalidades a un servicio web).
        \item \textbf{Fiabilidad}: La aplicación debe ser robusta y capaz de recuperarse de fallos.
        \item \textbf{Eficiencia}: La aplicación debe utilizar los recursos del sistema de manera eficiente.
    \end{itemize}
        
\end{document}