\documentclass{scrartcl}
\usepackage{graphicx}
\usepackage{float}
\usepackage[spanish]{babel}
\usepackage{hyperref} 
\setlength{\parskip}{\baselineskip}

\title{Elección de una Arquitectura}
\subtitle{\large PAMN - Programación de Aplicaciónes Moviles Nativas}
\author{Chamil José Cruz Razeq}

\begin{document}
    \maketitle
    \thispagestyle{empty}
    \newpage

    Todas las tareas propuestas se encuentran disponibles en el siguiente
         repositorio de \href{https://github.com/chamilstudy/ulpgc_pamn_labs}{GitHub}.

    \section{Supuesto 1}
        Al tratarse de un equipo muy pequeño, con poco presupuesto y tiempo de desarrollo,
         el modelo MVVM sería el idoneo por su facilidad de implementación. No obstante, 
         MVI ofrecería ciertas ventajas sobre todo en seguridad.
    \section{Supuesto 2}
        Al tratarse de un equipo pequeño, con poco tiempo de desrrollo donde premia la
         rapidez y la seguridad (al tratarse con información fragil como es la mensajería)
         un modelo idoneo sería el MVI, ya que facilita la creación de pruebas y plantea
         transferencia de información direccional. Además al tratarse de una aplicación social,
         el funcionamiento por "intención" justifica mas el uso de MVI.
    \section{Supuesto 3}
        Al tratarse de un software de largo desarrollo, para una gran empresa cuyas
         necesidades pueden cambiar o evolucionar con el tiempo, requieren de un modelo
         afín al cambio como plantea la arquitectura Hexagonal.
    \section{Supuesto 4}
        Al tratarse de un software de largo desarrollo, donde premia la estabilidad
         y la seguridad, MVI sería el modelo idoneo ya que nos permite la elaboración
         de pruebas con facilidad y plantea una transferencía de información direccional.
    \section{Supuesto 5}
        Al tratarse de un prototipo se optará por un desarrollo rápido que permita
         mostrar las funcionalidades básicas del software en un entorno controlado,
         por lo cual para lograr lo anterior, MVI sería el modelo idoneo para este
         cometido.
        

\end{document}
