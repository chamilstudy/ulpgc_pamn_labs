\documentclass{article}
\usepackage{graphicx}
\usepackage{float}
\usepackage[spanish]{babel}
\usepackage{hyperref}
\usepackage{csquotes}
\graphicspath{ {img/} }
\setlength{\parskip}{\baselineskip}

\title{Recomendaciones de Arquitectura para Aplicacioens Android \\[3ex] \small PAMN - Programación de Aplicaciónes Moviles Nativas}

\author{Chamil José Cruz Razeq}

\begin{document}
    \maketitle
    \thispagestyle{empty}
    \newpage

    \section{Introducción}
        Todos los informes sobre las tareas propuestas se encuentran disponibles en el
         siguiente repositorio de \href{https://github.com/chamilstudy/ulpgc_pamn_assigments}{GitHub}.

        Se realizará un breve estudio sobre las recomendaciones propuestas por "Google" \cite{recomendaciones}.
    
    \section{Recomendación 1: Diseño y Planificación}
        \subsection{Descripción}
        Consiste en garantizar una interfaz de usuario:
            \begin{itemize}
                \item Responsive, de forma que la aplicación de adecué a las dimensiones y tipo de 
                       dispositivo.
                \item Interactiva, aprovechando al máximo los recursos del dispositivo para un experiencia
                       de usuario rápida y eficaz. Para ello se aconseja fijar a 60 la tasa de fotogramas por
                       segundo y el uso de pantallas de carga para las pantallas o elementos que presenten mayor lentitud.
                \item Localización, ofreciendo idiomas alternativos y facilitando la traducción de la aplicación.
            \end{itemize}
            
        \subsection{Decisión}
        Se tendrán en cuenta estas recomendaciones al considerarse primordiales.

        \subsection{Justificación}
        El apartado visual y de experiencia de usuario es vital para el exito de una aplicación, si no se
         alcanzan unos mínimos de calidad, una mala presentación del producto puede ensombrecer el mejor de
         los back-ends.

    \section{Recomendación 2: Conectividad}
        \subsection{Descripción}
        Según "Google" más de la mitad de los usuario utilizan sus dispositivos a través de una
         red 2G, por lo que se debe tomar en consideración que datos se almacenarán localmente y
         cuales requerirán de conexión a internet.

        \subsection{Decisión}
        Se tendrá en cuenta esta recomendacón debido a la naturaleza de la aplicación a desarrollar.

        \subsection{Justificación}
        Dependiendo del tipo de aplicación a desarrollar y del público objetivo, puede interesarnos
         mas que la aplicación posea toda la información necesaria para su funcionamiento en el dispositivo,
         tal cómo es el caso de nuestra propuesta, siendo conscientes de que la aplicación podrá ser utilizada
         en localizaciones de dificil acceso a la red.

    \section{Recomendación 3: Pruebas Locales}
        \subsection{Descripción}
        Pruebas realizadas a traves del JVM y no desde el emulador, permitiendo realizarlas de manera más
         rapida y localizada.

        \subsection{Decisión}
        Se realizarán pruebas de este tipo.

        \subsection{Justificación}
        A la hora de probar bases de datos o el funcionamiento de ciertas funciones, estás pruebas son
         faciles de desarrollar y percibir.

    \section{Recomendación 4: Pruebas Instrumentadas}
        \subsection{Descripción}
        Pruebas realizadas en el propio dispositivo, físico o emulados.

        \subsection{Decisión}
        Se realizarán pruebas de este tipo.

        \subsection{Justificación}
        Estas pruebas son atractivas ya que se pueden observar los cambios en el propio programa y realizar
         pruebas tanto de interfaz como de carga de datos.

    \section{Recomendación 5: Privacidad}
        \subsection{Descripción}
        Se debe tener en cuenta los permisos que se solicitarán al usuario, siendo aconsejable minimizar la
         necesidad de estos, especialmente los de ubicación. Adicionalmente es importante informar a los
         usuarios sobre de que manera se estan utilizando sus datos y ofrecer la posibilidad de restablecerlos.

        \subsection{Decisión}
        Se tendrán en cuenta las recomendaciones.

        \subsection{Justificación}
        La integridad del software es vital para garantizar la confianza del usuario, no obstante el 
         software planteado no recopilará información delicada del usuario.

    \begin{thebibliography}{}
        \bibitem{recomendaciones} Recomendaciones de Arquitectura para Aplicaciones Android - https://developer.android.com/guide?hl=es-419
    \end{thebibliography}
        
\end{document}